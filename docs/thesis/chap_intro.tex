
\chapter{Introduction}
Long-endurance station-keeping aircraft are potential solutions to providing internet, communication, and persistent aerial coverage. 
Recently, tech companies such as Google\cite{googletitan} and Facebook\cite{aquila} have explored the possibility of using solar-electric powered aircraft to provide internet to parts of the world where 4G or 3G network is not available. 
Engineering firms like Aurora Flight Sciences\cite{orion} and Vanilla Aircraft\cite{vanilla} have developed gas powered long-endurance platforms as a means of providing continual surveillance for days at a time.  
Sizing of such long-endurance aircraft is complicated because of the multifaceted interaction between aerodynamics, structural weight, solar energy, wind speed, and other disciplines and environmental models.
The complexity of sizing these kinds of aircraft makes the trade offs between gas powered and solar-powered architectures potentially non-intuitive. 

This paper presents a physics-based optimization model that uses geometric programming as a systematic and rapid approach to evaluate trade offs between the two architectures for a given mission.  Geometric programming, a form of convex optimization, is chosen as a means of evaluating the design space because of its rapid solve time and guaranteed convergence to a global optimum.\cite{gp}
Models and equations representing the interaction of the various disciplines are expressed in a geometric programming form and are then combined to form an optimization model. 
The optimization can be solved in a fraction of a second and is used quantify to the feasible limits of the solar-electric and gas powered, long-endurance aircraft. 

The driving requirement that sizes solar-electric powered aircraft is the ability to operate at multiple locations and during all seasons.  
To fly multiple days, solar-electric powered aircraft must carry enough solar cells and batteries to fly during the day while storing enough energy to fly through the night.\cite{solartech}
If this condition can be achieved during the winter solstice, when the solar flux is at a minimum, then a solar-electric powered aircraft can theoretically fly during any time of the year for long durations.\cite{solartech}
This becomes more difficult to achieve at higher latitudes as solar flux during the winter solstice decreases.  
Additionally, to station keep the aircraft must fly faster than the local wind speeds.  
Wind speeds are a function of latitude, altitude, and season and tend to increase at higher latitudes and during winter months. 
Therefore, a key sizing study of solar-electric powered aircraft is the effect of latitude on aircraft size.  

The key driving requirement for a gas powered aircraft is endurance.  
As gas-powered aircraft are endurance limited by the amount of fuel that they can carry, longer endurance requires more fuel and hence a larger aircraft.  
Because gas-powered aircraft are not affected by the solar flux, their station-keeping ability at different latitudes only depends on the local wind speed. 
If a gas-powered aircraft can fly at the latitude with the worst wind speed, in can theoretically fly anywhere else in the world.  

Using this geometric programming methodology, trade offs between different aircraft configurations, power sources, and requirements can be calculated in seconds.  
The results show that gas-powered aircraft can generally be built lighter and can fly faster and can therefore fly at higher latitudes and in higher percentile wind speeds.  
Solar-powered aircraft have greater endurance but are limited operationally by their ability to reach higher speeds.  
The presented optimization methodology can quantify the difference in weight and endurance between and gas and solar-electric powered aircraft for the same set of requirements. 
Varying mission requirements reveals how the best architecture changes depending on the requirements. 


\chapter{Tail Boom Flexibility Forumlation}

I would like to thank Mark Drela for the formulation of the tail boom flexibility model.
The derivation of Equation~\eqref{e:smcorr} begins with the moment about the aircraft's center of mass

\begin{align}
    \label{e:mcenter}
    M_{cg} &= M_{\text{w}} + (x_{cg} - x_{ac})L_{\text{w}} - l_{\text{h}} L_{\text{h}} \\
    \label{e:eq6}
    \frac{M_{cg}}{qSc} = C_m &= C_{m_{\text{w}}} + \frac{x_{cg} - x_{ac}}{c} C_{L_W} - V_{\text{w}} C_{L_{\text{h}}}
\end{align}

It is assumed that the tail boom's effective root location is at the wing's aerodynamic center $x_{ac}$, and that the tail's pitching moment about its own aerodynamic center is negligible. 

Using moment and lift approximations

\begin{align}
    C_{m_{\text{w}}} &= constant \\
    C_{L_W} &= C_{L_{W_0}} + m_{\text{w}} \alpha \\
    \label{e:eq10}
    C_{L_{\text{h}}} &= C_{L_{h_0}} + m_{\text{h}} \left[\left( 1 - \frac{d\epsilon}{d\alpha}\right) - \theta \right] + C_{L_{h_{\delta}}}\delta_e \\
    m_{\text{w}} &= \frac{2\pi}{1 + 2/AR_{\text{w}}} \\
    m_{\text{h}} &= \frac{2\pi}{1 + 2/AR_{\text{h}}}
\end{align}

where $\epsilon$ is the wing's downwash angle seen at the tail. Using a vortex approximation and neglecting taper effects, we can estimate

\begin{equation}
    \frac{d\epsilon}{d\alpha} = \frac{m_{\text{w}}}{4\pi} \frac{c}{l_{\text{h}}}
\end{equation}

so that \eqref{e:eq10} can be written as

\begin{equation}
    C_{L_{\text{h}}} = m_{\text{h}} \left[\left( 1 - \frac{d\epsilon}{d\alpha}\right) - \frac{qS_{\text{h}}l_{\text{h}}^2}{EI_0}(1-\frac{1}{2}k) C_{L_{\text{h}}}\right] + C_{L_{h_{\delta}}}(\delta_e - \delta_{e_0}) 
\end{equation}

This can be further simplified by defining a tail boom flexibility factor $\mathcal{F}$.

\begin{align}
    C_{L_{\text{h}}} &= \mathcal{F}^{-1} m_{\text{h}} \left( 1 - \frac{d\epsilon}{d\alpha}\right) \alpha + \mathcal{F}^{-1} C_{L_{h_{\delta}}}(\delta_e - \delta_{e_0}) \\
    \mathcal{F} &= 1 + m_{\text{h}} \frac{qS_{\text{h}}l_{\text{h}}^2}{EI_0}(1-\frac{1}{2}k) 
\end{align}

Using the wing lift coefficient and the recast tail lift coefficient, the pitching moment is equation and derivative are given as follows.

\begin{align}
    C_m &= C_{m_{\text{w}}} + \frac{x_{cg} - x_{ac}}{c} (C_{L_{W_0}} + m_{\text{w}} \alpha) - V_{\text{w}} \left[ \mathcal{F}^{-1} m_{\text{h}} \left( 1 - \frac{d\epsilon}{d\alpha}\right) \alpha + \mathcal{F}^{-1} C_{L_{h_{\delta}}}(\delta_e - \delta_{e_0})\right] \\
    \frac{dC_m}{d\alpha} & = \frac{x_{cg} - x_{ac}}{c} m_{\text{w}}  - V_{\text{w}} \mathcal{F}^{-1} m_{\text{h}} \left( 1 - \frac{d\epsilon}{d\alpha}\right) 
\end{align}

Now by dividing by $dC_{L_W}/d\alpha = m_{\text{h}}$, which defines the static margin

\begin{equation}
    -\frac{dC_m/d\alpha}{dC_{L_W}/d\alpha} = \text{SM} = V_{\text{w}} \mathcal{F}^{-1} \frac{m_{\text{h}}}{m_{\text{w}}} \left( 1 - \frac{d\epsilon}{d\alpha}\right) - \frac{x_{cg} - x_{ac}}{c}
\end{equation}

The case for meeting the minimum static margin requirement is at the never-exceed dynamic pressure $q_{NE}$ and when the c.g is at its aft-most position.

\begin{equation}
    \text{SM}_{\text{min}} = V_{\text{w}} \mathcal{F}_{NE}^{-1} \frac{m_{\text{h}}}{m_{\text{w}}} \left( 1 - \frac{d\epsilon}{d\alpha}\right) - \frac{(x_{cg})_{\text{aft}} - x_{ac}}{c} 
\end{equation}

Dividing Equation~\eqref{e:eq6} by $C_{L_W}$, gives a requirement on the tail lift coefficient required to achieve a pitch trim condition $C_m=1$

\begin{equation}
    0 = \frac{C_{m_{\text{w}}}}{C_{L_W}} + \frac{x_{cg} - x_{ac}}{c} - V_{\text{w}} \frac{C_{L_{\text{h}}}}{C_{L_W}}
\end{equation}

The pitch authority requirement is that at the forward-most c.g. position and maximum lift with the tail lift coefficient equal to the most-negative allowable value $(C_{L_{\text{h}}})_{\text{min}}$. Combining the static margin and pitch authority requirement results in the horizontal tail sizing equation 

\begin{align}
    \text{SM}_{\text{min}} + \frac{\Delta x_{cg}}{c} - \frac{C_{M_{\text{w}}}}{C_{L_{\text{max}}}} &\leq V_{\text{w}} \mathcal{F}_{NE}^{-1} \frac{m_{\text{h}}}{m_{\text{w}}} \left( 1 - \frac{d\epsilon}{d\alpha}\right) + V_{\text{w}} \frac{-(C_{L_{\text{h}}})_{\text{min}}}{C_{L_{\text{max}}}} \\
\end{align}

where $\Delta x_{cg} = (x_{cg})_{\text{aft}} - (x_{cg})_{\text{fwd}}$.

% $Log: abstract.tex,v $
% Revision 1.1  93/05/14  14:56:25  starflt
% Initial revision
% 
% Revision 1.1  90/05/04  10:41:01  lwvanels
% Initial revision
% 
%
%% The text of your abstract and nothing else (other than comments) goes here.
%% It will be single-spaced and the rest of the text that is supposed to go on
%% the abstract page will be generated by the abstractpage environment.  This
%% file should be \input (not \include 'd) from cover.tex.
    Fueled by telecommunication needs and opportunities, there has been a recent push to develop aircraft that can provide long-endurance (days to weeks) persistent aerial coverage.
    These aircraft present a complicated systems engineering problem because of the multifaceted interaction between aerodynamics, structures, environmental effects, and engine, battery, and other component performance.
    Using geometric programming, models capturing the interaction between disciplines are used to analyze the feasible limits of solar-electric and gas powered, long-endurance aircraft in seconds to a level of detail and speed not previously achieved in initial aircraft sizing and design. 
    The results show that long-endurance, gas powered aircraft are generally more robust to higher wind speeds than solar-powered aircraft, but are limited in their endurance by the amount of fuel that they can carry. 
    While solar-electric powered aircraft can theoretically fly for months, they are operationally limited by reduced solar flux during the winter and wind speeds at higher latitudes.
    A detailed trade study between gas-powered and solar-powered aircraft is performed to discover which architecture is best suited to meet a given set of requirements, and what is the optimum size and endurance of that platform.
